\documentclass[]{beamer} % use setting 'handouts' to ignore overlay specifications, 'draft' to 
\usepackage[utf8]{inputenc}
\usetheme{Madrid} % beamer theme, can change
\usepackage{fancyvrb}
\usepackage{etoolbox}
\usepackage{tikz}
\usepackage[compat=1.1.0]{tikz-feynman}
\usepackage{ifluatex}

% example colors
\definecolor{vandy_gold}{RGB}{204, 161, 102}
\definecolor{vandy_blue}{RGB}{18, 111, 150}
\definecolor{vandy_red}{RGB}{153, 61, 27}
\definecolor{vandy_gray}{RGB}{51, 51, 51}
\definecolor{vandy_green}{RGB}{70, 78, 33}

% example beamer color settings
\setbeamercolor{section in toc}{fg=vandy_gold,bg=white}
\setbeamercolor{alerted text}{fg=vandy_red}
\setbeamercolor*{palette primary}{fg=white,bg=black}
\setbeamercolor*{palette secondary}{fg=white,bg=vandy_gold}
\setbeamercolor*{palette tertiary}{fg=white,bg=black}
\setbeamercolor*{palette quaternary}{fg=white,bg=vandy_gold}
\setbeamercolor*{sidebar}{fg=white,bg=black}
\setbeamercolor*{palette sidebar primary}{fg=white}
\setbeamercolor*{palette sidebar secondary}{fg=white}
\setbeamercolor*{palette sidebar tertiary}{fg=white}
\setbeamercolor*{palette sidebar quaternary}{fg=white}
\setbeamercolor{titlelike}{parent=palette secondary}
\setbeamercolor{frametitle}{fg=white,bg=vandy_gold}
\setbeamercolor{frametitle right}{fg=white,bg=vandy_gold}
\setbeamercolor{section number projected}{bg=black,fg=vandy_gold}
\setbeamercolor{section in toc}{fg=black}
\setbeamercolor{block title}{fg=white, bg=vandy_gray!50!white}
\setbeamercolor{block body}{bg=vandy_gray!20!white}

% example beamer format settings
\setbeamertemplate{itemize item}{\color{black}$\bullet$}
\setbeamertemplate{itemize subitem}{\color{black}$\bullet$}
\setbeamerfont{block title}{size=\small}
\setbeamerfont{block body}{size=\tiny}
\setbeamertemplate{itemize/enumerate body begin}{\tiny}
\setbeamertemplate{itemize/enumerate subbody begin}{\tiny}
\AtBeginEnvironment{tabular}{\tiny}
\tikzset{font=\tiny}
% \includeonlyframes{current} % if you want to just compile certain slides: use [current] setting on a frame to make it 'current'

% title slide information
\title[Short Name]{Full Name for Presentation}
\author[E. Sheridan]{\textbf{E. Sheridan}\inst{1}}
\institute{Vanderbilt University\inst{1}}

% useful commands:

% \uncover<+->{} takes overlay specifications and reveals contents accordingly
% \scalebox{x}{} scales contents by a factor of x

% document
\begin{document}

% title page
\frame{\titlepage}

% table of contents
\begin{frame}{Table of Contents}
\tableofcontents
\end{frame}

% example section
\section{Section 1}

% example frame 1
\begin{frame}{Example Frame}
    \begin{block}{Block Title}
        This block appears immediately
    \end{block}
    \begin{block}{Block Title}<1->
        This block appears on the second overlay and all following ones 
    \end{block}
\end{frame}

% example frame 2
\begin{frame}{Another Example Frame}
    \begin{block}{Block Title}<+->
        This block appears immediately
    \end{block}
    \begin{block}{Block Title}<+->
        This block appears next
    \end{block}
\end{frame}

% example section
\section{Section 2}

% example frame 3
\begin{frame}{Yet Another Example Frame}
    \begin{columns}
        \begin{column}{0.4\linewidth}
            \begin{block}{Block Title}<+(1)->
                This block appears almost immediately
            \end{block}
        \end{column}
        \begin{column}{0.4\linewidth}
            \begin{block}{Block Title}<+->
                This block appears next
            \end{block}
        \end{column}
    \end{columns}
\end{frame}

\end{document}