% PACKAGES

\usepackage[utf8]{inputenc}
\usepackage[dvipsnames]{xcolor}
\usepackage[margin=1.5in]{geometry}
\usepackage[english]{babel}
\usepackage{amsfonts, amsmath, amscd, amsmath, amsthm, amssymb}
\usepackage[mathscr]{eucal}
\usepackage{indentfirst} % indent first paragraph after headers
\usepackage{graphics, graphicx}
\usepackage{fancyhdr}  % headers and footers
%\usepackage{mathrsfs}
\usepackage{braket}  % bra-ket notation
\usepackage{array}
\usepackage{hyperref}
\usepackage{cleveref}
\usepackage{thmtools}
\usepackage[framemethod=TikZ]{mdframed}

% COMMANDS

\newcommand*\bell{\ensuremath{\boldsymbol\ell}}

\newcommand{\e}{\text{e}}
\renewcommand{\t}[1]{\text{#1}}
\renewcommand{\it}[1]{\textit{#1}}
\renewcommand{\bf}[1]{\textbf{#1}}
\renewcommand{\v}[1]{\mathbf{#1}}
\DeclareMathSymbol{\shortminus}{\mathbin}{AMSa}{"39}

\newcommand{\eps}{\varepsilon}

\newcommand{\st}{\textnormal{ such that }}
\newcommand{\for}{\textnormal{ for }}
\newcommand{\where}{\textnormal{ where }}
\newcommand{\by}{\textnormal{ by }}
\newcommand{\with}{\textnormal{ with }}
\newcommand{\ie}{\textnormal{ i.e. }}

\newcommand{\Rb}{\mathbb{R}}
\newcommand{\Rtb}{\mathbb{R}^3}
\newcommand{\Zb}{\mathbb{Z}}
\newcommand{\Nb}{\mathbb{N}}
\newcommand{\Qb}{\mathbb{Q}}
\newcommand{\Cb}{\mathbb{C}}
\newcommand{\Mb}{\mathbb{M}}

\newcommand{\Ac}{\mathcal{A}}
\newcommand{\Bc}{\mathcal{B}}
\newcommand{\Cc}{\mathcal{C}}
\newcommand{\Dc}{\mathcal{D}}
\newcommand{\Ec}{\mathcal{E}}
\newcommand{\Fc}{\mathcal{F}}
\newcommand{\Oc}{\mathcal{O}}
\newcommand{\Nc}{\mathcal{N}}
\newcommand{\Pc}{\mathcal{P}}
\newcommand{\Rc}{\mathcal{R}}
\newcommand{\Tc}{\mathcal{T}}
\newcommand{\Uc}{\mathcal{U}}

\DeclareMathOperator{\vspan}{span}
\DeclareMathOperator{\proj}{proj}
\DeclareMathOperator{\vol}{vol}
\DeclareMathOperator{\id}{id}
\DeclareMathOperator{\im}{im}
\renewcommand{\max}{\text{max}}
\renewcommand{\min}{\text{min}}

\newcommand{\wh}[1]{\widehat{#1}}
\newcommand{\wt}[1]{\widetilde{#1}}
\newcommand{\ol}[1]{\overline{#1}}
\newcommand{\ul}[1]{\underline{#1}}

\newcommand{\zeroton}{_{i=0}^n}
\newcommand{\zerotoi}{_{j=0}^i}
\newcommand{\oneton}{_{i=1}^n}
\newcommand{\zerotoinfty}{_{n=0}^{\infty}}
\newcommand{\onetoinfty}{ _{n=1} ^{\infty}}

\definecolor{gentlered}{HTML}{E76F64}
\newcommand{\vocab}[1]{\textbf{\color{gentlered} #1}}

% THEOREMS, ETC.: inspiration/format/basic syntax cite Evan Chen @ https://github.com/vEnhance/dotfiles/blob/master/texmf/tex/latex/evan/evan.sty

% MORAL
\newenvironment{moral}{%
	\begin{mdframed}[linecolor=gentlered]%
	\bfseries\color{gentlered}}%
	{\end{mdframed}
}

% DEFINITION
\definecolor{creme}{HTML}{F2D6B4}
\mdfdefinestyle{cremetab}{%
	skipabove=4pt,
	skipbelow=4pt,
	linewidth=2pt,
	rightline=false,
	leftline=true,
	topline=false,
	bottomline=false,
	linecolor=creme!60!Black,
	backgroundcolor=creme!20,
}
\declaretheoremstyle[
	headfont=\bfseries\sffamily\color{creme!60!Black},
	bodyfont=\normalfont,
	spaceabove=0.5pt,
	spacebelow=0.5pt,
	mdframed={style=cremetab},
	headpunct={ --- },
]{definition}

% DEFINITION 2
\mdfdefinestyle{cremerbox}{%
	roundcorner =10pt,
	linewidth=1pt,
	innerbottommargin=9pt,
	linecolor=creme!60!Black,
	nobreak=true,
	backgroundcolor=creme!20,
	skipabove=4pt,
	skipbelow=4pt,
}
\declaretheoremstyle[
	headfont=\sffamily\bfseries\color{creme!60!Black},
	mdframed={style=cremerbox},
	headpunct={\\[3pt]},
	postheadspace={0pt}
]{definition2}

% RESULT
\definecolor{softgreen}{HTML}{A9CBB8}
\mdfdefinestyle{greenrbox}{%
	roundcorner =10pt,
	linewidth=1pt,
	skipabove=4pt,
	skipbelow=4pt,
	innerbottommargin=9pt,
	linecolor=softgreen!80!black,
	nobreak=true,
	backgroundcolor=softgreen!10,
}
\declaretheoremstyle[
	headfont=\sffamily\bfseries\color{softgreen!80!black},
	mdframed={style=greenrbox},
	headpunct={\\[3pt]},
	postheadspace={0pt}
]{result}

% THEOREM
\definecolor{fgblue}{HTML}{7696BA}
\definecolor{bgblue}{HTML}{9DC6DD}
\mdfdefinestyle{bluetab}{%
	skipabove=4pt,
	skipbelow=4pt,
	linewidth=2pt,
	rightline=false,
	leftline=true,
	topline=false,
	bottomline=false,
	linecolor=fgblue,
	backgroundcolor=bgblue!12,
}
\declaretheoremstyle[
	headfont=\bfseries\sffamily\color{fgblue},
	bodyfont=\normalfont,
	spaceabove=0.5pt,
	spacebelow=0.5pt,
	mdframed={style=bluetab},
	headpunct={ --- },
]{theorem}

% COROLLARY
\definecolor{nicegray}{HTML}{B5B7AE}
\mdfdefinestyle{graytab}{%
	skipabove=4pt,
	skipbelow=4pt,
	linewidth=2pt,
	rightline=false,
	leftline=true,
	topline=false,
	bottomline=false,
	linecolor=nicegray!90!black,
	backgroundcolor=nicegray!10,
}
\declaretheoremstyle[
	headfont=\bfseries\sffamily\color{nicegray!90!black},
	bodyfont=\normalfont,
	spaceabove=0.5pt,
	spacebelow=0.5pt,
	mdframed={style=graytab},
	headpunct={ --- },
]{corollary}

% LEMMA
\definecolor{nicepink}{HTML}{FADADD}
\mdfdefinestyle{pinktab}{%
	skipabove=4pt,
	skipbelow=4pt,
	linewidth=2pt,
	rightline=false,
	leftline=true,
	topline=false,
	bottomline=false,
	linecolor=nicepink!80!black,
	backgroundcolor=nicepink!25,
}
\declaretheoremstyle[
	headfont=\bfseries\sffamily\color{nicepink!80!black},
	bodyfont=\normalfont,
	spaceabove=0.5pt,
	spacebelow=0.5pt,
	mdframed={style=pinktab},
	headpunct={ --- },
]{lemma}

% EXAMPLE
\definecolor{niceorange}{HTML}{EA9977}
\mdfdefinestyle{orangebox}{%
	linewidth=0.5pt,
	skipabove=4pt,
	skipbelow=4pt,
	frametitleaboveskip=5pt,
	frametitlebelowskip=0pt,
	frametitlefont=\bfseries,
	innertopmargin=4pt,
	innerbottommargin=8pt,
	nobreak=true,
	backgroundcolor=niceorange!5,
	linecolor=niceorange!85!black,
}
\declaretheoremstyle[
	headfont=\bfseries\color{niceorange!85!black},
	mdframed={style=orangebox},
	headpunct={\\[3pt]},
	postheadspace={0pt},
]{example}

% HW
\mdfdefinestyle{blackbox}{%
	linewidth=0.5pt,
	skipabove=4pt,
	skipbelow=4pt,
	frametitleaboveskip=5pt,
	frametitlebelowskip=0pt,
	frametitlefont=\bfseries,
	innertopmargin=4pt,
	innerbottommargin=8pt,
	nobreak=true,
	backgroundcolor=white,
	linecolor=black,
}
\declaretheoremstyle[
	headfont=\bfseries\color{black},
	mdframed={style=blackbox},
	headpunct={\\[3pt]},
	postheadspace={0pt},
]{homework}

\declaretheorem[style=definition2, name=Definition, numberwithin=subsection]{define}
\declaretheorem[style=result, name=Result, sibling=define]{res}
\declaretheorem[style=theorem, name=Theorem, sibling=define]{theo}
\declaretheorem[style=corollary, name=Corollary, sibling=define]{cor}
\declaretheorem[style=example, name=Example, sibling=define]{ex}
\declaretheorem[style=homework, name=Homework, sibling=define]{hw}
\declaretheorem[style=lemma, name=Lemma, sibling=define]{lem}

\Crefname{define}{Def.}{Defs.}
\Crefname{res}{Res.}{Res.}
\Crefname{theo}{Thm.}{Thms.}
\Crefname{cor}{Cor.}{Cors.}
\Crefname{ex}{Ex.}{Exs.}